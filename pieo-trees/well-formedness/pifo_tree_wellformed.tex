\documentclass{article}
\usepackage{graphicx} % Required for inserting images
\usepackage{amsmath}
\usepackage{amssymb}
\usepackage{geometry}
 \geometry{
 a4paper,
 top=25mm,
 bottom=25mm,
 }

\begin{document}

\section{Preservation of Well-Formedness In PIFO Trees}

\noindent This set of definitions and proofs completes the proof sketch outlined in Section 3.3, Definition 3.8 of Formal Abstractions In Packet Scheduling. In particular, it fleshes out the relationships between the functions push, pop and $|\cdot|$, in order to formally define (and by extension, prove) the preservation of well-formedness in PIFO Trees.\newline

\noindent This defines how pushing and popping modifies length in PIFOs, and supplies lemmas (with proof) clarifying how length changes with regards to pushing and popping in PIFO Trees. This then proves preservation of well-formedness in PIFO Trees following push and pop operations.\newline

\noindent This also lays some of the foundation for proving for well-formedness in PIEO Trees. Proofs of pushing translate almost one-to-one for PIEO Trees, while popping follows as an extension of this system, but with an inculsion of a totally ordered predicate.

\section{Relevant Definitions and Lemmas}

\subsection{Definition 1}

\noindent \textbf{\textit{Defining $|\cdot|$ on PIFOs With Regards to Pushes}}

\hfill \break 

\noindent \textbf{Definition 1A}: \textit{Pushing an element to a PIFO increments how many times it appears}
 $$\frac{p \in \text{PIFO}  \hspace{0.7cm} |p|_i = n \hspace{0.7cm} n \in \mathbb{N} \hspace{0.7cm} \text{PUSH}(p, i, r) = p' \hspace{0.7cm} n' = n+1}{|p'|_i = n'}$$

\noindent \textbf{Definition 1B:} \textit{Pushing an element to a PIFO does not change how much others appear}
$$\frac{p \in \text{PIFO}  \hspace{0.7cm} |p|_i = n \hspace{0.7cm} n \in \mathbb{N} \hspace{0.7cm} \text{PUSH}(p, j, r) = p' \hspace{0.7cm} i \neq j}{|p'|_i = n}$$

\noindent \textbf{Definition 1C:} \textit{Pushing an element to a PIFO increments its size}
$$\frac{p \in \text{PIFO}  \hspace{0.7cm} |p| = n \hspace{0.7cm} n \in \mathbb{N} \hspace{0.7cm} \text{PUSH}(p, i, r) = p' \hspace{0.7cm} n' = n+1}{|p'| = n'}$$\newline

\subsection{Definition 2}
\noindent \textbf{\textit{Defining $|\cdot|$ on PIFOs With Regards to Pops}}

\hfill\break

\noindent \textbf{Definition 2A}: \textit{Popping an element from a PIFO decrements how many times it appears}
 $$\frac{p \in \text{PIFO}  \hspace{0.7cm} |p|_i = n \hspace{0.7cm} n \in \mathbb{N} \hspace{0.7cm} \text{POP}(p) = (i, p') \hspace{0.7cm} n' = n-1}{|p'|_i = n'}$$

\noindent \textbf{Definition 2B:} \textit{Popping an element from a PIFO does not change how much others appear}
$$\frac{p \in \text{PIFO}  \hspace{0.7cm} |p|_i = n \hspace{0.7cm} n \in \mathbb{N} \hspace{0.7cm} \text{POP}(p) = (j, p') \hspace{0.7cm} i \neq j}{|p'|_i = n}$$

\noindent \textbf{Definition 2C:} \textit{Popping an element from a PIFO decrements its size}
$$\frac{p \in \text{PIFO}  \hspace{0.7cm} |p| = n \hspace{0.7cm} n \in \mathbb{N} \hspace{0.7cm} \text{POP}(p) = (i, p') \hspace{0.7cm} n' = n-1}{|p'| = n'}$$\newpage

\section{Lemmas 1 and 2}
\noindent \textbf{\textit{Defining $|\cdot|$ on PIFO Trees With Regards to Pushes and Pops}}\newline

\subsection{Lemma 1: Pushing to a PIFO Tree increments its size}

\par\

$$\frac{q \in \text{PIFOTree} \hspace{0.7cm} |q| = n \hspace{0.7cm} n \in \mathbb{N} \hspace{0.7cm} \text{push}(q, pkt, p) = q' \hspace{0.7cm} n' = n+1}{|q'| = n'}$$

\noindent \textbf{Proof}. We proceed upon this proof by structural induction over the definition of $|\cdot|$ as follows.\newline

\noindent \textbf{Base Case}: $q = \text{Leaf}(p)$ for some packet $p$.\newline

\noindent By definition 3.8, we have that $|\text{Leaf}(p)|_i = |p|_i$, where $|p|_i$ is the number of occurrences of $i$ in PIFO $p$. The definition of $\text{push}$ in 3.6 gives the following:

$$\frac{\text{PUSH}(p, pkt, r) = p'}{\text{push}(\text{Leaf}(p), pkt, r) = \text{Leaf}(p')}$$\\[-10pt]

\noindent Where $\text{Leaf}(p')$ is the result of $\text{push}$ing into $\text{Leaf}(p)$.\newline

\noindent By definition 1A, we now know that $|p'|_i = |p|_i + 1$.\newline

\noindent By definition 3.8, we know that $|\text{Leaf}(p')|_i = |p'|_i$ and $|\text{Leaf}(p)|_i = |p|_i$.\newline

\noindent Combining these together, we obtain the following:

$$|\text{Leaf}(p')| = |\text{Leaf}(p)| + 1$$

\noindent Thus, we have proven our base case.\\[10pt]

\noindent \textbf{Inductive Case}:\newline

\noindent Assume an arbitrary PIFO $p$, Node $q = \text{Internal}(qs, p)$, packet $pkt$, path $pt$, index $i$ and rank $r$.\newline

\noindent \textbf{Inductive Hypothesis}: $|qs[i]| = n \implies |\text{push}(qs[i], pkt, pt)| = n+1$\newline

\noindent \textbf{Show}: $|\text{Internal}(qs, p)| = m \implies |\text{push}(\text{Internal}(qs, p), pkt, (i, r) :: pt)| = m+1$\newline

\noindent Note that $\text{push}$ is defined for internal nodes as follows:

$$\frac{\text{push}(qs[i], pkt, pt) = q' \hspace{1cm} \text{PUSH}(p, i, r) = p'}{\text{push}(\text{Internal}(qs, p), pkt, (i, r) :: pt) = \text{Internal}(qs[q'/i], p')}$$\\[-10pt]

\noindent Since $q$ is a PIFO Tree, we know from Definition 3.8 the following:

$$m = |q| = \sum_{j=1}^{|qs|} |qs[j]|$$

\noindent Thus, we can further conclude the following:

$$|\text{push}(\text{Internal}(qs, p), pkt, (i, r) :: pt)| = \sum_{j=1}^{|qs|} |qs[q'/i][j]|$$

\noindent From our Inductive Hypothesis, we know that $|q'| = |\text{push}(qs[i], pkt, pt| = n+1$.\newline

\noindent From the definition of $\text{push}$, pushing changes only $qs[i]$, while the remaining subtrees remain the same. Subsequently, we can assert that $\forall 
1 \leq j \leq |qs|, i \neq j \implies |qs[j]| = |qs[q'/i][j]$. With this in mind, we make the following claim:\\[-15pt]

$$|\text{push}(\text{Internal}(qs, p), pkt, (i, r) :: pt)| = \sum_{j=1}^{|qs|} |qs[j]| - |qs[i]| + |qs[q'/i][i]|$$

\noindent Substituting what we know, this gives us the following:

$$|\text{push}(\text{Internal}(qs, p), pkt, (i, r) :: pt)| = m - n + n + 1$$

$$\implies |\text{push}(\text{Internal}(qs, p), pkt, (i, r) :: pt)| = m + 1$$\\[-15pt]

\noindent With this, we have shown the desired equality, and proven our Inductive case.\newline

\noindent Thus, it follows that Lemma 1 holds.\newline

\subsection{Lemma 2: Popping from a PIFO Tree decrements its size}

\par\

$$\frac{q \in \text{PIFOTree} \hspace{0.7cm} |q| = n \hspace{0.7cm} n \in \mathbb{N} \hspace{0.7cm} (pkt, q') = \text{pop}(q) \hspace{0.7cm} n' = n-1}{|q'| = n'}$$

\noindent \textbf{Proof}. We proceed upon this proof by structural induction over the definition of $|\cdot|$ as follows.\newline

\noindent \textbf{Base Case}: $q = \text{Leaf}(p)$ for some packet $p$.\newline

\noindent By definition 3.8, we have that $|\text{Leaf}(p)|_i = |p|_i$, where $|p|_i$ is the number of occurrences of $i$ in PIFO $p$. The definition of $\text{pop}$ in 3.6 gives the following:

$$\frac{\text{POP}(p) = p'}{\text{pop}(\text{Leaf}(p)) = \text{Leaf}(pkt, p')}$$\\[-10pt]

\noindent Where $\text{Leaf}(p')$ is the result of $\text{pop}$ping from $\text{Leaf}(p)$.\newline

\noindent By definition 2A, we now know that $|p'|_i = |p|_i - 1$.\newline

\noindent By definition 3.8, we know that $|\text{Leaf}(p')|_i = |p'|_i$ and $|\text{Leaf}(p)|_i = |p|_i$.\newline

\noindent Combining these together, we obtain the following:

$$|\text{Leaf}(p')| = |\text{Leaf}(p)| - 1$$

\noindent Thus, we have proven our base case.\\[10pt]

\noindent \textbf{Inductive Case}:\newline

\noindent Assume an arbitrary PIFO $p$, Node $q = \text{Internal}(qs, p)$ and index $i$.\newline

\noindent \textbf{Inductive Hypothesis}: $|qs[i]| = n \implies |q'| = n-1$ where $(pkt, q') = \text{pop}(qs[i])$.\newline

\noindent \textbf{Show}: $|\text{Internal}(qs, p)| = m \implies |q''| = m-1$ where $(pkt, q'') =\text{pop}(\text{Internal}(qs, p)$.\newline

\noindent Note that $\text{pop}$ is defined for internal nodes as follows:

$$\frac{\text{POP(p) = (i, p') \hspace{1cm} \text{pop}(qs[i]) = (pkt, q')}}{{\text{pop}(\text{Internal}(qs, p)) = (pkt, \text{Internal}(qs[q'/i], p'))}}$$\\[-10pt]

\noindent Since $q$ is a PIFO Tree, we know from Definition 3.8 the following:

$$m = |q| = \sum_{j=1}^{|qs|} |qs[j]|$$

\noindent Thus, we can further conclude the following:

$$(pkt, q'') = \text{pop}(\text{Internal}(qs, p) \implies |q''| = \sum_{j=1}^{|qs|} |qs[q'/i][j]|$$

\noindent From our Inductive Hypothesis, we know that $|q'| = n-1$ where $pkt, q' = \text{pop}(qs[i])$.\newline

\noindent From the definition of $\text{pop}$, popping changes only $qs[i]$, while the remaining subtrees remain the same. Subsequently, we can assert that $\forall 
1 \leq j \leq |qs|, i \neq j \implies |qs[j]| = |qs[q'/i][j]$. With this in mind, we make the following claim:\\[-15pt]

$$(pkt, q'') = \text{pop}(\text{Internal}(qs, p)) \implies |q''| = \sum_{j=1}^{|qs|} |qs[j]| - |qs[i]| + |qs[q'/i][i]|$$

\noindent Substituting what we know, this gives us the following:

$$(pkt, qs'') = \text{pop}(\text{Internal}(qs, p)) \implies |q''| = m - n + n - 1$$

$$\implies (pkt, qs'') = \text{pop}(\text{Internal}(qs, p)) \implies |qs''| = m - 1$$\\[-15pt]

\noindent With this, we have shown the desired equality, and proven our Inductive case.\newline

\noindent Thus, it follows that Lemma 2 holds.\newline


\section{Proofs For Lemma 3.9 (Well-Formedness)}

\subsection{Proof of Lemma 3.9.1}
\noindent \textit{\textbf{Well-Formedness Is Preserved in PIFO Trees Upon Pushes}}\\[10pt]

\noindent We proceed with this proof by inducting upon the definition of $\text{push}$ for PIFO Trees.\newline

\noindent \textbf{Base Case : $Leaf(p)$}\newline

\noindent By definition 3.6 of $\text{push}$, we have the following for Leaf nodes:

$$\frac{\text{PUSH}(p, pkt, r) = p'}{\text{push}(\text{Leaf}(p), pkt, r) = \text{Leaf}(p')}$$\\[-10pt]

\noindent It follows that after pushing, the resultant tree is expressed as $\text{Leaf}(p')$ for some packet $p'$ By definition 3.8 (cited above), we know the following holds for any arbitrary PIFO $p$:\\[-10pt]

$$\frac{}{\vdash \text{Leaf}(p)}$$

\noindent We now have: $\forall p, pkt, r, \vdash \text{push}(\text{Leaf}(p), pkt, r)$\newline

\noindent With this, our Base Case is proven.\\[10pt]

\noindent \textbf{Inductive Case:} $q = \text{Internal}(qs, p), \vdash q$\newline

\noindent \textbf{Inductive Hypothesis}: $\forall 1 \leq i \leq |qs|. \vdash \text{push}(qs[i], pkt, pt)$\newline

\noindent Recall Definition 3.8 for $\vdash$ on Internal Nodes:

$$\frac{\forall 1 \leq i \leq |qs|. \vdash qs[i] \land |p|_i = |qs[i]|}{\vdash \text{Internal}(\text{qs, p})}$$\newline

\noindent Recall Definition 3.6 for $\text{push}$ on Internal Nodes:

$$\frac{\text{push}(qs[i], pkt, pt) = q' \hspace{1cm} \text{PUSH}(p, i, r) = p'}{\text{push}(\text{Internal}(qs, p), pkt, (i, r) :: pt) = \text{Internal}(qs[q'/i], p')}$$\newline

\noindent Now, let $q' = \text{push}(qs[i], pkt, pt)$, and let $p' = \text{PUSH}(p, i, r)$.\newline

\noindent \textbf{Show}: $\vdash q \implies \vdash \text{Internal}(qs[q'/i], p')$.\newline

\noindent Using definition 3.6 and our Inductive Hypothesis, we can conclude that $\vdash q'$. Furthermore, note that the only subtree to be modified in $qs$ is in $qs[i]$, per the definition of $\text{push}$.\newline

\noindent \textbf{Now we make the following claim:}\\[-20pt]

\begin{alignat*}{5}
&(1) \vdash \text{Internal}(qs, p) && \hspace{0.3cm} \textit{By definition}\\[10pt]
&(2) \implies \forall 1 \leq j \leq |qs|. \hspace{0.2cm} \vdash qs[j] \land |p|_j = |qs[j]|&& \hspace{0.3cm} \textit{Inversion Lemma (Definition 3.8)}\\[5pt]
&(3) \implies |p|_i = |qs[i]|&& \hspace{0.3cm} \textit{Instance of universal quantifier (2)}\\[5pt]
&(4) \indent |p'|_i = |p|_i + 1&& \hspace{0.3cm} \textit{Definition 1A}\\[5pt]
&(5) \indent |q'| = |qs[i]| + 1&& \hspace{0.3cm} \textit{Lemma 1}\\[5pt]
&(6) \implies |p|_i + 1 = |qs[i]| + 1&& \hspace{0.3cm} \textit{Addition Property of Equality}\\[5pt]
&(7) \implies |p'|_i = |q'|&& \hspace{0.3cm} \textit{Substitution (4, 5, 6)}\\[5pt]
&(8) \indent \forall 1 \leq j \leq |p'|, i \neq j \implies |p'|_j = |p|_j&& \hspace{0.3cm} \textit{Definition 1B}\\[5pt]
&(9) \implies \forall 1 \leq j \leq |p'|, i \neq j \implies |p'|_j = |qs[j]|&& \hspace{0.3cm} \textit{By (2) and (8)}\\[5pt]
&(10) \implies \forall 1 \leq j \leq |p'|, |p'|_j = |qs[q' / i][j]|&& \hspace{0.3cm} \textit{By (7) and (9)}\\[5pt]
&(11)  \indent \forall 1 \leq j \leq |qs|. \vdash qs[q'/i][j]&& \hspace{0.3cm} \textit{Inductive Hypothesis}\\[5pt]
&(12) \implies \forall 1 \leq j \leq |qs|. \vdash qs[q'/i][j] \land |p'|_j = |qs[q'/i][j]|&&\hspace{0.3cm} \textit{By (10) and (11)}\\[5pt]
&(13) \implies \vdash \text{Internal}(qs[q'/i], p') && \hspace{0.3cm} \textit{Definition of } \vdash\\[-10pt]
\end{alignat*}

\noindent With this, we have proven our Inductive Statement and completed the proof. We have shown that pushing to an arbitrary PIFO Tree preserves its well-formedness.\\[-10pt]

\subsection{Proof of Lemma 3.9.2}
\noindent \textit{\textbf{Well-Formedness Is Preserved in PIFO Trees Upon Pops}}\\[5pt]

\noindent We proceed with this proof by inducting upon the definition of $\text{pop}$ for PIFO Trees.\newline

\noindent \textbf{Base Case : $Leaf(p)$}\newline

\noindent By definition 3.4 of $\text{pop}$, the following holds for Leaf nodes:

$$\frac{\text{POP}(p) = (pkt, p')}{\text{pop}(\text{Leaf}(p)) = (pkt, \text{Leaf}(p'))}$$\\[-15pt]

\noindent It follows that after popping, the resultant tree is of form $\text{Leaf}(p')$ for some packet $p'$. By definition 3.8 (cited above), we know the following holds for any arbitrary PIFO $p$:\\[-20pt]

$$\frac{}{\vdash \text{Leaf}(p)}$$\\[-10pt]

\noindent We now have: $\forall p, \vdash p'$ where $(pkt, p') = \text{pop}(\text{Leaf}(p))$.\newline

\noindent With this, our Base Case is proven.\newline


\noindent \textbf{Inductive Case:} $q = \text{Internal}(qs, p), \vdash q$\newline

\noindent \textbf{Inductive Hypothesis}: $\forall 1 \leq j \leq |qs|. \vdash qs[j]',$ where $(pkt[j], qs[j]') = \text{pop}(qs[j])$\newline

\noindent Recall Definition 3.8 for $\vdash$ on Internal Nodes:

$$\frac{\forall 1 \leq i \leq |qs|. \vdash qs[i] \land |p|_i = |qs[i]|}{\vdash \text{Internal}(\text{qs, p})}$$\\[-10pt]

\noindent Recall Definition 3.6 for $\text{pop}$ on Internal Nodes:

$$\frac{\text{POP(p) = (i, p') \hspace{1cm} \text{pop}(qs[i]) = (pkt, q')}}{{\text{pop}(\text{Internal}(qs, p)) = (pkt, \text{Internal}(qs[q'/i], p'))}}$$\\[-10pt]

\noindent Now, let $(pkt, q') = \text{pop}(qs[i])$, and let $(i, p') = \text{POP}(p)$.\newline

\noindent \textbf{Show}: $\vdash q \implies \vdash \text{Internal}(qs[q'/i], p')$.\newline

\noindent Using definition 3.6 and our Inductive Hypothesis, we can conclude that $\vdash q'$. Furthermore, note that the only subtree to be modified in $qs$ is in $qs[i]$, per the definition of $\text{pop}$.\newline

\noindent \textbf{Now we make the following claim:}\\[-20pt]

\begin{alignat*}{5}
&(1) \vdash \text{Internal}(qs, p) && \hspace{0.3cm} \textit{By definition}\\[10pt]
&(2) \implies \forall 1 \leq j \leq |qs|. \hspace{0.2cm} \vdash qs[j] \land |p|_j = |qs[j]|&& \hspace{0.3cm} \textit{Inversion Lemma (Definition 3.8)}\\[5pt]
&(3) \implies |p|_i = |qs[i]|&& \hspace{0.3cm} \textit{Instance of universal quantifier (2)}\\[5pt]
&(4) \indent |p'|_i = |p|_i - 1&& \hspace{0.3cm} \textit{Definition 1A}\\[5pt]
&(5) \indent |q'| = |qs[i]| - 1&& \hspace{0.3cm} \textit{Lemma 1}\\[5pt]
&(6) \implies |p|_i - 1 = |qs[i]| - 1&& \hspace{0.3cm} \textit{Subtraction Property of Equality}\\[5pt]
&(7) \implies |p'|_i = |q'|&& \hspace{0.3cm} \textit{Substitution (4, 5, 6)}\\[5pt]
&(8) \indent \forall 1 \leq j \leq |p'|, i \neq j \implies |p'|_j = |p|_j&& \hspace{0.3cm} \textit{Definition 1B}\\[5pt]
&(9) \implies \forall 1 \leq j \leq |p'|, i \neq j \implies |p'|_j = |qs[j]|&& \hspace{0.3cm} \textit{By (2) and (8)}\\[5pt]
&(10) \implies \forall 1 \leq j \leq |p'|, |p'|_j = |qs[q' / i][j]|&& \hspace{0.3cm} \textit{By (7) and (9)}\\[5pt]
&(11)  \indent \forall 1 \leq j \leq |qs|. \vdash qs[q'/i][j]&& \hspace{0.3cm} \textit{Inductive Hypothesis}\\[5pt]
&(12) \implies \forall 1 \leq j \leq |qs|. \vdash qs[q'/i][j] \land |p'|_j = |qs[q'/i][j]|&&\hspace{0.3cm} \textit{By (10) and (11)}\\[5pt]
&(13) \implies \vdash \text{Internal}(qs[q'/i], p') && \hspace{0.3cm} \textit{Definition of } \vdash\\[-10pt]
\end{alignat*}

\noindent With this, we have proven our Inductive Statement and completed the proof. We have shown that popping from an arbitrary PIFO Tree preserves its well-formedness.\\[-10pt]

\end{document}

