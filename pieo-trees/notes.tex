\documentclass{amsart}

% font
\usepackage{cmbright}

% margin 
\usepackage[margin=1in]{geometry}

% references
\usepackage[colorlinks]{hyperref} 
\PassOptionsToPackage{colorlinks}{hyperref}
\hypersetup{urlcolor = RedViolet, linkcolor = RoyalBlue, citecolor = NavyBlue}
\urlstyle{same}

\usepackage{amsmath, amssymb, amsthm} 
\usepackage[svgnames, dvipsnames]{xcolor}
\usepackage{mhsetup, mathtools}

\usepackage[capitalise]{cleveref}
\usepackage[T1]{fontenc} 
\usepackage{silence} % for suppressing warnings
\usepackage{cite}

% PL macros
\usepackage{mathpartir}
\usepackage{stmaryrd}
\newcommand{\inference}[3]{\inferrule*[Right=#1]{#2}{#3}}
\newcommand{\axiom}[2]{\inferrule*[Right=#1]{\;}{#2}}
\DeclareMathOperator{\bigstep}{\Downarrow}
\DeclareMathOperator{\false}{\textbf{false}}
\DeclareMathOperator{\true}{\mathbf{true}}
\newcommand{\hoare}[3]{\{#1\} \; #2 \; \{#3\}}

% (Formal Abstractions macros)++
\DeclareMathOperator{\halfto}{\rightharpoonup}
\DeclareMathOperator{\pkt}{\mathrm{pkt}}
\DeclareMathOperator{\push}{\mathrm{push}}
\DeclareMathOperator{\pop}{\mathrm{pop}}
\DeclareMathOperator{\Pkt}{\mathbf{Pkt}}
\DeclareMathOperator{\Rk}{\mathbf{Rk}}
\DeclareMathOperator{\Data}{\mathbf{Data}}
\DeclareMathOperator{\Topo}{\mathbf{Topo}}
\DeclareMathOperator{\Path}{\mathbf{Path}}
\DeclareMathOperator{\PIEO}{\mathbf{PIEO}}
\DeclareMathOperator{\PIFO}{\mathbf{PIFO}}
\DeclareMathOperator{\PIEOTree}{\mathbf{PIEOTree}}
\DeclareMathOperator{\PIFOTree}{\mathbf{PIFOTree}}
\DeclareMathOperator{\Leaf}{\mathrm{Leaf}}
\DeclareMathOperator{\Internal}{\mathrm{Internal}}
\DeclareMathOperator{\Node}{\mathrm{Node}}

% theorems
\newtheorem{thm}{Theorem}[section]
\newtheorem{lem}[thm]{Lemma}
\theoremstyle{definition}
\newtheorem{dfn}[thm]{Definition}
\newtheorem{abuse}[thm]{Abuse of Notation}
\newtheorem{ex}[thm]{Example}

% no more indent
\setlength{\parindent}{0pt}

% right-justified sections hack
\usepackage{titlesec}
\newcommand{\marginsecnumber}[1]{%
  \makebox[0pt][r]{#1\hspace{6pt}}%
}
\titleformat{\section}
  {\normalfont\Large\bfseries}
  {\marginsecnumber\thesection}
  {0pt}
  {}
\titleformat{\subsection}
  {\normalfont\large\bfseries}
  {\marginsecnumber\thesubsection}
  {0pt}
  {}
\titleformat{\subsubsection}
  {\normalfont\normalsize\bfseries}
  {\marginsecnumber\thesubsubsection}
  {0pt}
  {}

\begin{document}

\pagestyle{empty}

{\LARGE \textbf{PIEO Trees for Fun and Profit}}

\hrulefill\\

We assume familiarity with \cite{OG}, adopt its notational conventions, and borrow many of its definitions!

\section{Structure \& Semantics}

\begin{dfn}
    For sets $S$, $D$, and predicates $F$ over $D$,
    let $\PIEO(S, D, F)$ denote the set of \emph{PIEO}s that
    \begin{enumerate}
        \item hold entries in $S$, decorated with meta-data in $D$
        \item are ordered by $\Rk$
        \item support predicates in $F$
        \item admit partial functions 
        \begin{align*}
            \pop &: \PIEO(S, D, F) \times F \halfto S \times \PIEO(S, D, F) \\
            \push &: \PIEO(S, D, F) \times S \times D \times \Rk \to \PIEO(S, D, F) 
        \end{align*}
    \end{enumerate}
    For $p \in \PIEO(S, D, F)$, $s \in S$, and $f \in F$, we write 
    \begin{enumerate}
        \item $|p|$ for the number of entries in $p$
        \item $|p|_{s}$ for the number of times $s$ occurs in $p$
        \item $|p|_{s,f}$ for the number of times $s$ occurs in $p$ with associated $d \in D$ such that $f(d)$ holds
    \end{enumerate}
\end{dfn}

We fix an opaque set $\Data$ and a collection $\mathcal F$ of predicates defined on it.
These predicates come with a total order $\leq$ and the property that, $\forall d \in \Data$ and $f,f^\prime \in \mathcal F$,
$f \leq f^\prime \land f(d) \implies f^\prime(d)$.

\begin{dfn}
    The set of \emph{PIEO trees} over $t \in \Topo$, denoted $\mathbf{PIEOTree}(t)$, is defined inductively by
    \begin{align*}
        \inference{}
        {
            p \in \mathbf{PIEO}(\Pkt, \Data, \mathcal F)
        }
        {
            \Leaf(p) \in \mathbf{PIEOTree}(\ast)
        }
        &&
        \inference{}
        {
            n \in \mathbb N\\
            ts \in \mathbf{Topo}^n\\
            p \in \mathbf{PIEO}(\{1,\ldots, n\}, \mathbf{Data}, \mathcal F)\\\\
            \forall i \in [1, n]. \; qs[i] \in \mathbf{PIEOTree}(ts[i])
        }
        {
            \Internal(qs, p) \in \mathbf{PIEO}(\Node(ts))
        }
    \end{align*}
\end{dfn}

\begin{dfn}
    Define $\pop : \PIEOTree(t) \times \mathcal F \halfto \Pkt \times \PIEOTree(t)$ by
    \begin{align*}
        \inference{}
        {
            \pop(p, f) = (\pkt, p^\prime)
        }
        {
            \pop(\Leaf(p), f) = (\pkt, \Leaf(p^\prime))
        }
        &&
        \inference{}
        {
            \pop(p, f) = (i, p^\prime)\\
            \pop(qs[i], f) = (\pkt, q^\prime)
        }
        {
            \pop(\Internal(qs, p), f) = (\pkt, \Internal(qs[q^\prime/i], p^\prime))
        }
    \end{align*}
\end{dfn}

\begin{dfn}
    Define $\push : \PIEOTree(t) \times \Pkt \times \Data \times \Path(t) \to \PIEOTree(t)$ by
    \begin{align*}
        \inference{}
        {
            \push(p, \pkt, d, r) = p^\prime
        }
        {
            \push(\Leaf(p), \pkt, d, r) = \Leaf(p^\prime)
        }
        &&
        \inference{}
        {
            \push(p, i, d, r) = p^\prime\\ 
            \push(qs[i],  \pkt, d, pt) = q^\prime
        }
        {
            \push(\Internal(qs, p), \pkt, d, (i, r) :: pt) = \Internal(qs[q^\prime/i], p^\prime)
        }
    \end{align*}
\end{dfn}

\begin{dfn}
    Let $t \in \Topo$. 
    A \emph{control} over $t$ is a triple $(s, q, z)$, where $s \in \mathrm{St}$ is the \emph{current state},
    $q$ is a PIEO tree of topology $t$, 
    and $z : \mathrm{St} \times \Pkt \to \Data \times \Path(t) \times \mathrm{St}$ is a function called the \emph{scheduling transaction}.
    We write $\mathrm{Control}(t)$ for the set of controls over $t$.
\end{dfn}

\begin{dfn}
    Define $|\cdot| : \PIEOTree(t) \to \mathbb N$ by
    \begin{align*}
        |\Leaf(p)| = |p| && |\Internal(qs, p)| = \sum_{i=1}^{|qs|} |qs[i]|
    \end{align*}

    We say that $q \in \PIEOTree(t)$ is \emph{well-formed} w.r.t $f \in \mathcal F$, denoted $\vdash_f q$, if it adheres to the following rules.
    \begin{align*}
        \axiom{}
        {\vdash_f \Leaf(p)}
        &&
        \inference{}
        {
            \forall i \in [1, |qs|] \; \vdash_f qs[i] \land |p|_i = |qs[i]|
        }
        {
            \vdash_f \Internal(qs, p)
        }
    \end{align*}

    We say $q$ is well-formed, denoted $\vdash q$, if there exists $f \in \mathcal F$ such that, for all $f^\prime \geq f$, $\vdash_{f^\prime} q$.
\end{dfn}

\section{Embedding and Simulation}

\begin{dfn}
    Let $t_1, t_2 \in \Topo$. 
    We call a relation $R \subseteq \PIFOTree(t_1) \times \PIFOTree(t_2)$ a \emph{simulation} if, 
    for all $\pkt \in \Pkt$, $f \in \mathcal F$, and $q_1 \; R \; q_2$,
    \begin{enumerate}
        \item If $\pop(q_1, f)$ is undefined, then so is $\pop(q_2, f)$
        \item If $\pop(q_1, f) = (\pkt, q_1^\prime)$, then $\pop(q_2) = (\pkt, q_2^\prime)$ such that $q_1^\prime \; R \; q_2^\prime$.
        \item For all $pt_1 \in \Path(t_1)$ and $d \in \Data$, there exists $pt_2 \in \Path(t_2)$ such that
            $$\push(q_1, \pkt, d, pt_1) \; R \; \push(q_2, \pkt, d, pt_2)$$
    \end{enumerate}
\end{dfn}

\newpage 

\renewcommand\refname{\LARGE References}
\bibliographystyle{alpha} 
\bibliography{refs}

\end{document}
