\documentclass{amsart}

% font
\usepackage{cmbright}

% margin 
\usepackage[margin=1in]{geometry}

% references
\usepackage[colorlinks]{hyperref} 
\PassOptionsToPackage{colorlinks}{hyperref}
\hypersetup{urlcolor = RedViolet, linkcolor = RoyalBlue, citecolor = NavyBlue}
\urlstyle{same}

\usepackage{amsmath, amssymb, amsthm} 
\usepackage[svgnames, dvipsnames]{xcolor}
\usepackage{mhsetup, mathtools}

\usepackage[capitalise]{cleveref}
\usepackage[T1]{fontenc} 
\usepackage{silence} % for suppressing warnings
\usepackage{cite}

% PL macros
\usepackage{mathpartir}
\usepackage{stmaryrd}
\newcommand{\inference}[3]{\inferrule*[Right=#1]{#2}{#3}}
\newcommand{\axiom}[2]{\inferrule*[Right=#1]{\;}{#2}}
\DeclareMathOperator{\bigstep}{\Downarrow}
\DeclareMathOperator{\false}{\textbf{false}}
\DeclareMathOperator{\true}{\mathbf{true}}
\newcommand{\hoare}[3]{\{#1\} \; #2 \; \{#3\}}

% Formal Abstractions macros
\DeclareMathOperator{\pkt}{\mathrm{pkt}}

% theorems
\newtheorem{thm}{Theorem}[section]
\newtheorem{lem}[thm]{Lemma}
\theoremstyle{definition}
\newtheorem{dfn}[thm]{Definition}
\newtheorem{abuse}[thm]{Abuse of Notation}
\newtheorem{ex}[thm]{Example}

% no more indent
\setlength{\parindent}{0pt}

% right-justified sections hack
\usepackage{titlesec}
\newcommand{\marginsecnumber}[1]{%
  \makebox[0pt][r]{#1\hspace{6pt}}%
}
\titleformat{\section}
  {\normalfont\Large\bfseries}
  {\marginsecnumber\thesection}
  {0pt}
  {}
\titleformat{\subsection}
  {\normalfont\large\bfseries}
  {\marginsecnumber\thesubsection}
  {0pt}
  {}
\titleformat{\subsubsection}
  {\normalfont\normalsize\bfseries}
  {\marginsecnumber\thesubsubsection}
  {0pt}
  {}

\begin{document}

\pagestyle{empty}

{\LARGE \textbf{PIEO Trees for Fun and Profit}}

\hrulefill\\

We assume familiarity with \cite{OG}, adopt its notational conventions, and borrow many of its definitions!

% \section{Overview}
% 
% {
%     \color{red}
%     TODO!
%     \begin{enumerate}
%         \item An example of a scheduling transaction that cannot be expressed with PIFO trees: maybe token bucket?
%         \item Introduce what a PIEO is and what it does 
%         \item An overview of what to expect with each section.
%     \end{enumerate}
% }
% 
% \newpage

\section{Structure \& Semantics}

Our PIEO trees are near identical to PIFO trees, with only two modifications:
\begin{enumerate}
    \item All nodes hold PIEOs instead of PIFOs
    \item Internal PIEOs hold just enough extra data to compute predicates of packets that live in subtrees.
\end{enumerate}

To make ``just enough extra data'' precise, let $f$ be a predicate on $\mathbf{Pkt}$. 
We'll presuppose the existence of a set $\mathbf{Data}_f$ and associated surjective map $\mathrm{data}_f: \mathbf{Pkt} \to \mathbf{Data}_f$.
This map and set are such that, for $\pkt, \pkt^\prime \in \mathbf{Pkt}$, 
$$\mathrm{data}_f(\pkt) = \mathrm{data}_f(\pkt^\prime) \implies f(\pkt) = f(\pkt^\prime)$$
In other words, all preimages of $d \in \mathbf{Data}_f$ evaluate to the same truth value under $f$. 
For this reason,

\begin{abuse}
    We'll often write $f(d)$ when we mean $f(\pkt)$ for some $\pkt \in f^{-1}(d)$. 
\end{abuse}

Broadly, this set $\mathbf{Data}_f$ is designed to encode just the parts of $\mathbf{Pkt}$ necessary for computing $f(\pkt)$.

\begin{ex}
    For any $f$, it is legal to set $\mathrm{Data}_f = \mathbf{Pkt}$ and $\mathrm{data}_f$ to be the identity map.
    We'd expect this to work since each $\pkt \in \mathbf{Pkt}$ certainly holds all the information necessary to compute $f(\pkt)$.
\end{ex}

We can now formally define PIEO trees!

\begin{dfn}[PIEO tree]
    Let $t \in \mathbf{Topo}$ be a topology and $f$ a predicate over $\mathbf{Pkt}$.
    The set of \emph{PIEO trees} over $t$ and $f$, denoted $\mathbf{PIEOTree}(t, f)$, is defined inductively by
    \begin{align*}
        \inference{}
        {
            p \in \mathbf{PIEO}(\mathbf{Pkt}, f)
        }
        {
            \mathrm{Leaf}(p) \in \mathbf{PIEOTree}(\ast)
        }
        &&
        \inference{}
        {
            n \in \mathbb N\\
            ts \in \mathbf{Topo}^n\\
            p \in \mathbf{PIEO}(\{1,\ldots, n\} \times \mathbf{Data}_f(\mathbf{Pkt}), f)\\\\
            \forall i \in [1, n]. \; qs[i] \in \mathbf{PIEOTree}(ts[i])
        }
        {
            \mathrm{Internal}(qs, p) \in \mathbf{PIEO}(\mathrm{Node}(ts))
        }
    \end{align*}
\end{dfn}

\newpage 

\renewcommand\refname{\LARGE References}
\bibliographystyle{alpha} 
\bibliography{refs}

\end{document}
